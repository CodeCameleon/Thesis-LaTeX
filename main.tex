\documentclass[12pt]{report} % A betűméret 12-re és a dokumentum típus jelentés-re állítása.

\usepackage[magyar]{babel} % A magyar nyelvi csomag betöltése.
\usepackage[T1]{fontenc} % A kimeneti karakterkódolás T1-re állítása.
\usepackage{times} % A Times New Roman betűtípus beállítása.
\usepackage[utf8]{inputenc} % A karakterkódolás UTF8-ra állítása.
\usepackage{fancyhdr} % A fejlécekhez és a láblécekhez használt csomag.
\usepackage{graphicx} % A képek beillesztéséhez szükséges csomag.
\usepackage{titlesec} % A címformázáshoz szükséges csomag.
\usepackage{parskip} % Kikapcsolja az automatikus paragrafus bekezdéseket.

\hoffset -1in % A dokumentum vízszintes eltolása 1 inch-tel balra.
\voffset -1in % A dokumentum függőleges eltolása 1 inch-tel felfelé.
\oddsidemargin 35mm % A páratlan oldalak bal margójának beállítása 35 mm-re.
\textwidth 150mm % A szöveg szélességének beállítása 150 mm-re.
\topmargin 15mm % A felső margó beállítása 15 mm-re.
\headheight 10mm % A fejléc magasságának beállítása 10 mm-re.
\headsep 5mm % A fejléc és a szöveg közötti távolság beállítása 5 mm-re.
\textheight 237mm % A szöveg magasságának beállítása 237 mm-re.

\begin{document} % A dokumentum kezdete.

% A fejezetek formázásának módosítása.
\titleformat{\chapter}[hang] % Ez biztosítja a hagyományos elrendezést.
  {\normalfont\huge\bfseries} % A fejezet címe nagy méretű, félkövér betűtípussal jelenik meg.
  {\thechapter.}{1em}{} % A fejezet száma ponttal elválasztva jelenik meg, a cím közvetlenül utána következik.

% A fejezetek első oldalának stílusának testreszabása.
\fancypagestyle{plain}{
    \fancyhf{} % Törli az alapértelmezett fejlécet és láblécet.
    \fancyfoot[R]{\raisebox{4mm}\thepage} % A lap számát a lábléc jobb oldalra rakja és feljebb tolja.
    \renewcommand{\headrulewidth}{0pt} % Törli a fejléc vonalát.
}

\pagestyle{fancy} % A többi oldalnak a stílusának testreszabása.
\fancyhf{} % Törli az alapértelmezett fejlécet és láblécet.
\fancyhead[L]{Titkosítási módszerek a jövő blokkláncai számára} % A fejléc bal oldalára a diplomamunka címének elhelyezése.
\fancyfoot[R]{\raisebox{4mm}\thepage} % A lap számát a lábléc jobb oldalra rakja és feljebb tolja.

\thispagestyle{empty} % Erről az oldalról törli a fejlécet és láblécet.

\begin{center} % Vízszintesen középre igazítja a tartalmat.
    {\Large\bf Szegedi Tudományegyetem}\\ % Nagy és félkövér szöveg.
    \vspace{0.5cm} % Függőleges térköz beszúrása.
    {\Large\bf Informatikai Intézet}\\ % Nagy és félkövér szöveg.
    \vspace*{8.5cm} % Függőleges térköz kötelező beszúrása.
    {\Huge\bf SZAKDOLGOZAT}\\ % Óriási és félkövér szöveg.
    \vspace*{7cm} % Függőleges térköz kötelező beszúrása.
    {\Large\bf Molnár Gábor Ádám}\\ % Nagy és félkövér szöveg.
    \vspace*{0.6cm} % Függőleges térköz kötelező beszúrása.
    {\Large\bf 2025} % Nagy és félkövér szöveg.
\end{center} % A középre igazított blokk vége.

\newpage % Új oldalt kezd.

\thispagestyle{empty} % Erről az oldalról törli a fejlécet és láblécet.

\begin{center} % Vízszintesen középre igazítja a tartalmat.
    \vspace*{1cm} % Függőleges térköz kötelező beszúrása.
    {\Large\bf Szegedi Tudományegyetem}\\ % Nagy és félkövér szöveg.
    \vspace{0.5cm} % Függőleges térköz beszúrása.
    {\Large\bf Informatikai Intézet}\\ % Nagy és félkövér szöveg.
    \vspace*{3.8cm} % Függőleges térköz kötelező beszúrása.
    {\LARGE\bf Titkosítási módszerek a jövő blokkláncai számára}\\ % Nagy és félkövér szöveg.
    \vspace*{3.6cm} % Függőleges térköz kötelező beszúrása.
    {\Large Szakdolgozat}\\ % Nagy szöveg.
    \vspace*{4cm} % Függőleges térköz kötelező beszúrása.
    {\large % Nagy szöveg.
    \begin{tabular}{c@{\hspace{4cm}}c} % Két oszlop 4 cm-es térközzel.
    \emph{Készítette:} &\emph{Témavezető:}\\ % Dőlt címkék.
    \textbf{Molnár Gábor Ádám} &\textbf{Dr. habil. Kertész Attila}\\ % Félkövér nevek.
    informatika szakos &egyetemi docens\\ % Szak és beosztás kiírása.
    hallgató& % Az utolsó sorban nincs második oszlopbeli tartalom.
    \end{tabular}
    }\\
    \vspace*{2.3cm} % Függőleges térköz kötelező beszúrása.
    {\Large % Nagy szöveg.
    Szeged % Szeged kiírása.
    \\ % Új sor
    \vspace{2mm} % Függőleges térköz beszúrása.
    2025 % Az évszám kiírása.
    }
\end{center} % A középre igazított blokk vége.

\tableofcontents % Tartalomjegyzék beszúrása

\chapter*{Feladatkiírás} % Sorszám nélküli fejezet létrehozása.
\addcontentsline{toc}{chapter}{Feladatkiírás} % A fejezet hozzáadása a tartalomjegyzékhez.

A felhő és blokklánc-alapú platformok iránti érdeklődés jelentősen megnőtt, ami az ellátási láncok racionalizálásában, a nyomon követhetőség javításában, a kereskedelem egyszerűsítésében és a pénzügyi tranzakciók hatékonyabbá tételében rejlő lehetőségüknek köszönhető. Ez a fellendülés a Bitcoin egyre népszerűvé válásával kezdődött, rávilágítva a régebbi blokklánc platformok energiafogyasztással és késleltetéssel kapcsolatos kihívásaira.

A modern blokklánc platformok ezen problémák megoldását célozták meg, és hatékonyabb módszereket nyújtanak a különböző üzleti alkalmazásoknak.

A blokklánc rendszerek egy fontos eleme a titkosítási algoritmusok alkalmazása. Napjainkban az SHA-256 algoritmust használják a tranzakciók és a hálózati kommunikáció biztonságossá tételére, de a szakirodalomban számos egyéb algoritmust is javasoltak bizonyos tulajdonságok javítására (például titkosítás erőssége, gyorsasága).

A szakdolgozó célja ezen szakirodalmi megoldások áttekintése, tanulmányozása, különös tekintettel a jövő blokklánc rendszerei és poszt-kvantum megoldásai tekintetében. További cél egy saját titkosítási algoritmus kifejlesztése és összehasonlítása teljesítménymérések elvégzésével, a szakirodalomban elérhető közeli megoldásokkal összevetve.

\chapter*{Tartalmi összefoglaló} % Sorszám nélküli fejezet létrehozása.
\addcontentsline{toc}{chapter}{Tartalmi összefoglaló} % A fejezet hozzáadása a tartalomjegyzékhez.

\section*{A téma megnevezése} % Sorszám nélküli alcím létrehozása.

Titkosítási módszerek a jövő blokkláncai számára és ezek összehasonlítása.

\section*{A megadott feladat megfogalmazása} % Sorszám nélküli alcím létrehozása.

A feladat a különbező titkosítási algoritmusok összehasonlítás, különböző szempontok alapján. Továbbá egy saját algoritmus fejlesztése és összehasonlítás a már létező és aktívan használt algoritmusokkal.

\section*{Megoldási mód} % Sorszám nélküli alcím létrehozása.

Egy grafikus felülettel rendelkező alkalmazás elkészítése, amelyen könnyedén lehet futtatni Unit teszteket a különböző algoritmusokra. Valamint alkalmas az eredmények áttekinthető kimutatására.

\section*{Alkalmazott eszközök, módszerek} % Sorszám nélküli alcím létrehozása.

Az alkalmazás egy ASP.NET Core webalkalmazás lett. A tesztek futtatására az NUnit csomagot használja, a megjelenítésre pedig a Chart.js csomagot. Az eredményeket egy SQLite adatbázisban tárolja.

\section*{Elért eredmények} % Sorszám nélküli alcím létrehozása.

Egy teljesen működő és tesztelésre alkalmas webalkalmazás. Továbbá egy saját szimmetrikus titkosítási algoritmus C\# megvalósítása.

\section*{Kulcsszavak} % Sorszám nélküli alcím létrehozása.

C\#, ASP.NET Core, NUnit, Chart.js, blokkláncok, kriptográfia

\chapter{Bevezetés} % Sorszámozott fejezet létrehozása.

\section{Motiváció} % Sorszámozott alcím létrehozása.

\section{Kriptográfia} % Sorszámozott alcím létrehozása.

\section{Poszt-kvantum kriptográfia} % Sorszámozott alcím létrehozása.

\section{Blokkláncok} % Sorszámozott alcím létrehozása.

\chapter{Felhasznált technológiák} % Sorszámozott fejezet létrehozása.

\section{C\#} % Sorszámozott alcím létrehozása.

\section{ASP.NET Core} % Sorszámozott alcím létrehozása.

\section{NUnit} % Sorszámozott alcím létrehozása.

\section{Chart.js} % Sorszámozott alcím létrehozása.

\section{BouncyCastle.Cryptography} % Sorszámozott alcím létrehozása.

\section{Git, GitHub, GitLab} % Sorszámozott alcím létrehozása.

\chapter{Architekturális terv} % Sorszámozott fejezet létrehozása.

\section{Webalkalmazás} % Sorszámozott alcím létrehozása.

\section{Model-View-Controller} % Sorszámozott alcím létrehozása.

\section{Szeparáció} % Sorszámozott alcím létrehozása.

\section{Adatbázis} % Sorszámozott alcím létrehozása.

\section{Algoritmusok egységesítés} % Sorszámozott alcím létrehozása.

\section{Tesztelt tulajdonságok} % Sorszámozott alcím létrehozása.

\section{Eredmények megjelenítése} % Sorszámozott alcím létrehozása.

\chapter{Megvalósítás} % Sorszámozott fejezet létrehozása.

\section{Az alkalmazás alapjai} % Sorszámozott alcím létrehozása.

\section{Algoritmusok} % Sorszámozott alcím létrehozása.

\section{NUnit tesztek} % Sorszámozott alcím létrehozása.

\section{Tesztesetek automatizálása} % Sorszámozott alcím létrehozása.

\section{Adattárolás} % Sorszámozott alcím létrehozása.

\section{A webalkalmazás} % Sorszámozott alcím létrehozása.

\section{Tesztek integrálása} % Sorszámozott alcím létrehozása.

\section{Eredmények megjelenítése} % Sorszámozott alcím létrehozása.

\chapter{Saját algoritmus} % Sorszámozott fejezet létrehozása.

\section{Elképzelés} % Sorszámozott alcím létrehozása.

\section{A fejlődés folyamata} % Sorszámozott alcím létrehozása.

\section{Jelenlegi állapot} % Sorszámozott alcím létrehozása.

\chapter{Teszteredmények} % Sorszámozott fejezet létrehozása.

\section{Szimmetrikus algoritmusok} % Sorszámozott alcím létrehozása.

\section{Aszimmetrikus algoritmusok} % Sorszámozott alcím létrehozása.

\section{Hasító algoritmusok} % Sorszámozott alcím létrehozása.

\section{A saját algoritmusom} % Sorszámozott alcím létrehozása.

\chapter{Összefoglalás} % Sorszámozott fejezet létrehozása.

\chapter*{Irodalomjegyzék} % Sorszámozott fejezet létrehozása.
\addcontentsline{toc}{chapter}{Irodalomjegyzék} % A fejezet hozzáadása a tartalomjegyzékhez.

\chapter*{Nyilatkozat} % Sorszámozott fejezet létrehozása.
\addcontentsline{toc}{chapter}{Nyilatkozat} % A fejezet hozzáadása a tartalomjegyzékhez.

Alulírott Molnár Gábor Ádám, programtervező informatikus BSc szakos hallgató, kijelentem, hogy a dolgozatomat a Szegedi Tudományegyetem, Informatikai Intézet Szoftverfejlesztés Tanszékén készítettem, programtervező informatikus BSc diploma megszerzése érdekében.

Kijelentem, hogy a dolgozatot más szakon korábban nem védtem meg, saját munkám eredménye, és csak a hivatkozott forrásokat (szakirodalom, eszközök stb.) használtam fel.

Tudomásul veszem, hogy szakdolgozatomat a Szegedi Tudományegyetem Diplomamunka Repozitóriumában tárolja.

\vspace{1cm}

{\large Kelt.: 2025. május 13.}

\vspace{0.5cm}
\hfill
\parbox{5cm}{\centering\hrule\vspace{0.3cm} Aláírás}

\end{document} % A dokumentum vége.
