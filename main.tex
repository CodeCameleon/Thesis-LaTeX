\documentclass[12pt]{report} % A betűméret 12-re és a dokumentum típus jelentés-re állítása.

\usepackage[magyar]{babel} % A magyar nyelvi csomag betöltése.
\usepackage[T1]{fontenc} % A kimeneti karakterkódolás T1-re állítása.
\usepackage{times} % A Times New Roman betűtípus beállítása.
\usepackage[utf8]{inputenc} % A karakterkódolás UTF8-ra állítása.
\usepackage{fancyhdr} % A fejlécekhez és a láblécekhez használt csomag.
\usepackage{graphicx} % A képek beillesztéséhez szükséges csomag.
\usepackage{indentfirst} % Az első bekezdés automatikus behúzásához szükséges csomag.

\hoffset -1in % A dokumentum vízszintes eltolása 1 inch-tel balra.
\voffset -1in % A dokumentum függőleges eltolása 1 inch-tel felfelé.
\oddsidemargin 35mm % A páratlan oldalak bal margójának beállítása 35 mm-re.
\textwidth 150mm % A szöveg szélességének beállítása 150 mm-re.
\topmargin 15mm % A felső margó beállítása 15 mm-re.
\headheight 10mm % A fejléc magasságának beállítása 10 mm-re.
\headsep 5mm % A fejléc és a szöveg közötti távolság beállítása 5 mm-re.
\textheight 237mm % A szöveg magasságának beállítása 237 mm-re.

\begin{document} % A dokumentum kezdete.

% A fejezetek első oldalának stílusának testreszabása.
\fancypagestyle{plain}{
    \fancyhf{} % Törli az alapértelmezett fejlécet és láblécet.
    \fancyfoot[R]{\raisebox{5mm}\thepage} % A lap számát a lábléc jobb oldalra rakja és 5mm-el feljebb tolja.
    \renewcommand{\headrulewidth}{0pt} % Törli a fejléc vonalát.
}

\pagestyle{fancy} % A többi oldalnak a stílusának testreszabása.
\fancyhf{} % Törli az alapértelmezett fejlécet és láblécet.
\fancyhead[L]{Titkosítási módszerek a jövő Blokkláncai számára} % A fejléc bal oldalára a diplomamunka címének elhelyezése.
\fancyfoot[R]{\raisebox{5mm}\thepage} % A lap számát a lábléc jobb oldalra rakja és 5mm-el feljebb tolja.

\thispagestyle{empty} % Erről az oldalról törli a fejlécet és láblécet.

\begin{center} % Vízszintesen középre igazítja a tartalmat.
    \vspace*{1cm} % Függőleges térköz kötelező beszúrása.
    {\Large\bf Szegedi Tudományegyetem}\\ % Nagy és félkövér szöveg.
    \vspace{0.5cm} % Függőleges térköz beszúrása.
    {\Large\bf Informatikai Intézet}\\ % Nagy és félkövér szöveg.
    \vspace*{3.8cm} % Függőleges térköz kötelező beszúrása.
    {\LARGE\bf Titkosítási módszerek a jövő Blokkláncai számára}\\ % Nagy és félkövér szöveg.
    \vspace*{3.6cm} % Függőleges térköz kötelező beszúrása.
    {\Large Szakdolgozat}\\ % Nagy szöveg.
    \vspace*{4cm} % Függőleges térköz kötelező beszúrása.
    {\large % Nagy szöveg.
    \begin{tabular}{c@{\hspace{4cm}}c} % Két oszlop 4 cm-es térközzel.
    \emph{Készítette:} &\emph{Témavezető:}\\ % Dőlt címkék.
    \textbf{Molnár Gábor Ádám} &\textbf{Dr. Kertész Attila}\\ % Félkövér nevek.
    informatika szakos &egyetemi docens\\ % Szak és beosztás kiírása.
    hallgató& % Az utolsó sorban nincs második oszlopbeli tartalom.
    \end{tabular}
    }\\
    \vspace*{2.3cm} % Függőleges térköz kötelező beszúrása.
    {\Large % Nagy szöveg.
    Szeged % Szeged kiírása.
    \\ % Új sor
    \vspace{2mm} % Függőleges térköz beszúrása.
    2025 % Az évszám kiírása.
    }
\end{center} % A középre igazított blokk vége.

\tableofcontents % Tartalomjegyzék beszúrása

\chapter*{Feladatkiírás} % Sorszám nélküli fejezet létrehozása.
\addcontentsline{toc}{chapter}{Feladatkiírás} % A fejezet hozzáadása a tartalomjegyzékhez.

A felhő és blokklánc-alapú platformok iránti érdeklődés jelentősen megnőtt, ami az ellátási láncok racionalizálásában, a nyomon követhetőség javításában, a kereskedelem egyszerűsítésében és a pénzügyi tranzakciók hatékonyabbá tételében rejlő lehetőségüknek köszönhető. Ez a fellendülés a Bitcoin egyre népszerűvé válásával kezdődött, rávilágítva a régebbi blokklánc platformok energiafogyasztással és késleltetéssel kapcsolatos kihívásaira.
\vspace{0.5cm}

A modern blokklánc platformok ezen problémák megoldását célozták meg, és hatékonyabb módszereket nyújtanak a különböző üzleti alkalmazásoknak.
\vspace{0.5cm}

A blokklánc rendszerek egy fontos eleme a titkosítási algoritmusok alkalmazása. Napjainkban az SHA-256 algoritmust használják a tranzakciók és a hálózati kommunikáció biztonságossá tételére, de a szakirodalomban számos egyéb algoritmust is javasoltak bizonyos tulajdonságok javítására (pl. titkosítás erőssége, gyorsasága).
\vspace{0.5cm}

A szakdolgozó célja ezen szakirodalmi megoldások áttekintése, tanulmányozása, különös tekintettel a jövő blokklánc rendszerei és poszt-kvantum megoldásai tekintetében. További cél egy saját titkosítási algoritmus kifejlesztése és összehasonlítása teljesítménymérések elvégzésével, a szakirodalomban elérhető közeli megoldásokkal összevetve.

\chapter*{Tartalmi összefoglaló} % Sorszám nélküli fejezet létrehozása.
\addcontentsline{toc}{chapter}{Tartalmi összefoglaló} % A fejezet hozzáadása a tartalomjegyzékhez.

\end{document} % A dokumentum vége.
