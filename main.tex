\documentclass[12pt]{report} % A betűméret 12-re és a dokumentum típus jelentés-re állítása.

\def\magyarOptions{defaults=hu-min} % Konfigurálja a magyar nyelvi beállításokat minimális szabályokkal.

\usepackage[magyar]{babel} % A magyar nyelvi csomag betöltése.
\usepackage[T1]{fontenc} % A kimeneti karakterkódolás T1-re állítása.
\usepackage{times} % A Times New Roman betűtípus beállítása.
\usepackage[utf8]{inputenc} % A karakterkódolás UTF8-ra állítása.
\usepackage{fancyhdr} % A fejlécekhez és a láblécekhez használt csomag.
\usepackage{graphicx} % A képek beillesztéséhez szükséges csomag.
\usepackage{titlesec} % A címformázáshoz szükséges csomag.
\usepackage{parskip} % Kikapcsolja az automatikus paragrafus bekezdéseket.
\usepackage{setspace} % A sortávolság beállításához szükséges csomag.
\usepackage{xcolor} % A színezéshez szükséges csomag.
\usepackage{csquotes} % A nyelvfüggő szövegekhez szükséges csomag.
\usepackage[hidelinks]{hyperref} % A linkekhez szükséges csomag.
\usepackage[backend=biber, style=numeric, sorting=none]{biblatex} % Az irodalmak számozott hivatkozásához szükséges csomag.
\usepackage[a4paper, left=25mm, top=25mm, right=25mm, bottom= 25mm]{geometry} % PDF-hez igazított, szimmetrikus 2,5 cm margók minden oldalon.

\addbibresource{references.bib} % Az irodalomjegyzék forrásainak megadása.

\renewcommand*{\bibfont}{\normalfont\fontsize{10}{12}\selectfont} % A betűméret 10-re, a sormagasságot pedig 12-re állítja az irodalomjegyzékben.

\setlength\bibitemsep{\baselineskip} % Az irodalomjegyzék sortávolságának beállítása.

% A magyar irodalomjegyzék beállítása.
\DefineBibliographyStrings{magyar}{
  url = {Elérhetőség:},
  urlseen = {Utolsó megtekintés dátuma:},
}

\DeclareFieldFormat{url}{\\\bibstring{url}\addcolon\space\href{#1}{\textcolor{blue}{\underline{#1}}}} % Megformázza az URL-t az irodalomjegyzékben.

\DeclareFieldFormat{urldate}{\\\bibstring{urlseen}\addcolon\space#1} % Megformázza a megtekintés dátumát az irodalomjegyzékben.

\onehalfspacing % A másfeles sortávolság beállítása.

\setlength{\headheight}{14.5pt} % Növeljük a fejléc magasságát.

\begin{document} % A dokumentum kezdete.

% A fejezetek formázásának módosítása.
\titleformat{\chapter}[hang] % Ez biztosítja a hagyományos elrendezést.
  {\normalfont\huge\bfseries} % A fejezet címe nagy méretű, félkövér betűtípussal jelenik meg.
  {\thechapter.}{1em}{} % A fejezet száma ponttal elválasztva jelenik meg, a cím közvetlenül utána következik.

\titlespacing*{\chapter}{0pt}{-20pt}{\baselineskip} % A fejezetek előtti és utáni térközök beállítása.

% A fejezetek első oldalának stílusának testreszabása.
\fancypagestyle{plain}{
    \fancyhf{} % Törli az alapértelmezett fejlécet és láblécet.
    \fancyhead[L]{Titkosítási módszerek a jövő blokkláncai számára} % A fejléc bal oldalára a diplomamunka címének elhelyezése.
    \fancyfoot[R]{\thepage} % A lap számát a lábléc jobb oldalra rakja.
}

\pagestyle{fancy} % A többi oldalnak a stílusának testreszabása.
\fancyhf{} % Törli az alapértelmezett fejlécet és láblécet.
\fancyhead[L]{Titkosítási módszerek a jövő blokkláncai számára} % A fejléc bal oldalára a diplomamunka címének elhelyezése.
\fancyfoot[R]{\thepage} % A lap számát a lábléc jobb oldalra rakja.

\thispagestyle{empty} % Erről az oldalról törli a fejlécet és láblécet.

\begin{center} % Vízszintesen középre igazítja a tartalmat.
    {\Large\bf Szegedi Tudományegyetem}\\ % Nagy és félkövér szöveg.
    \vspace{0.5cm} % Függőleges térköz beszúrása.
    {\Large\bf Informatikai Intézet}\\ % Nagy és félkövér szöveg.
    \vspace*{8.5cm} % Függőleges térköz kötelező beszúrása.
    {\Huge\bf SZAKDOLGOZAT}\\ % Óriási és félkövér szöveg.
    \vspace*{10cm} % Függőleges térköz kötelező beszúrása.
    {\Large\bf Molnár Gábor Ádám}\\ % Nagy és félkövér szöveg.
    \vspace*{0.6cm} % Függőleges térköz kötelező beszúrása.
    {\Large\bf 2025} % Nagy és félkövér szöveg.
\end{center} % A középre igazított blokk vége.

\newpage % Új oldalt kezd.

\thispagestyle{empty} % Erről az oldalról törli a fejlécet és láblécet.

\begin{center} % Vízszintesen középre igazítja a tartalmat.
    \vspace*{1cm} % Függőleges térköz kötelező beszúrása.
    {\Large\bf Szegedi Tudományegyetem}\\ % Nagy és félkövér szöveg.
    \vspace{0.5cm} % Függőleges térköz beszúrása.
    {\Large\bf Informatikai Intézet}\\ % Nagy és félkövér szöveg.
    \vspace*{3.8cm} % Függőleges térköz kötelező beszúrása.
    {\LARGE\bf Titkosítási módszerek a jövő blokkláncai számára}\\ % Nagy és félkövér szöveg.
    \vspace*{3.6cm} % Függőleges térköz kötelező beszúrása.
    {\Large Szakdolgozat}\\ % Nagy szöveg.
    \vspace*{4cm} % Függőleges térköz kötelező beszúrása.
    {\large % Nagy szöveg.
    \begin{tabular}{c@{\hspace{4cm}}c} % Két oszlop 4 cm-es térközzel.
    \emph{Készítette:} &\emph{Témavezető:}\\ % Dőlt címkék.
    \textbf{Molnár Gábor Ádám} &\textbf{Dr. habil. Kertész Attila}\\ % Félkövér nevek.
    informatika szakos &egyetemi docens\\ % Szak és beosztás kiírása.
    hallgató& % Az utolsó sorban nincs második oszlopbeli tartalom.
    \end{tabular}
    }\\
    \vspace*{2.5cm} % Függőleges térköz kötelező beszúrása.
    {\Large % Nagy szöveg.
    Szeged % Szeged kiírása.
    \\ % Új sor
    \vspace{2mm} % Függőleges térköz beszúrása.
    2025 % Az évszám kiírása.
    }
\end{center} % A középre igazított blokk vége.

\tableofcontents % Tartalomjegyzék beszúrása

\chapter*{Feladatkiírás} % Sorszám nélküli fejezet létrehozása.
\addcontentsline{toc}{chapter}{Feladatkiírás} % A fejezet hozzáadása a tartalomjegyzékhez.

A felhő és blokklánc-alapú platformok iránti érdeklődés jelentősen megnőtt, ami az ellátási láncok racionalizálásában, a nyomon követhetőség javításában, a kereskedelem egyszerűsítésében és a pénzügyi tranzakciók hatékonyabbá tételében rejlő lehetőségüknek köszönhető. Ez a fellendülés a Bitcoin egyre népszerűvé válásával kezdődött, rávilágítva a régebbi blokklánc platformok energiafogyasztással és késleltetéssel kapcsolatos kihívásaira.

A modern blokklánc platformok ezen problémák megoldását célozták meg, és hatékonyabb módszereket nyújtanak a különböző üzleti alkalmazásoknak.

A blokklánc rendszerek egy fontos eleme a titkosítási algoritmusok alkalmazása. Napjainkban az SHA-256 algoritmust használják a tranzakciók és a hálózati kommunikáció biztonságossá tételére, de a szakirodalomban számos egyéb algoritmust is javasoltak bizonyos tulajdonságok javítására (például titkosítás erőssége, gyorsasága).

A szakdolgozó célja ezen szakirodalmi megoldások áttekintése, tanulmányozása, különös tekintettel a jövő blokklánc rendszerei és poszt-kvantum megoldásai tekintetében. További cél egy saját titkosítási algoritmus kifejlesztése és összehasonlítása teljesítménymérések elvégzésével, a szakirodalomban elérhető közeli megoldásokkal összevetve.

\chapter*{Tartalmi összefoglaló} % Sorszám nélküli fejezet létrehozása.
\addcontentsline{toc}{chapter}{Tartalmi összefoglaló} % A fejezet hozzáadása a tartalomjegyzékhez.

\section*{A téma megnevezése} % Sorszám nélküli alcím létrehozása.

Titkosítási módszerek a jövő blokkláncai számára és ezek összehasonlítása.

\section*{A megadott feladat megfogalmazása} % Sorszám nélküli alcím létrehozása.

A feladat a különbező titkosítási algoritmusok összehasonlítás, különböző szempontok alapján. Továbbá egy saját algoritmus fejlesztése és összehasonlítás a már létező és aktívan használt algoritmusokkal.

\section*{Megoldási mód} % Sorszám nélküli alcím létrehozása.

Egy grafikus felülettel rendelkező alkalmazás elkészítése, amelyen könnyedén lehet futtatni Unit teszteket a különböző algoritmusokra. Valamint alkalmas az eredmények áttekinthető kimutatására.

\section*{Alkalmazott eszközök, módszerek} % Sorszám nélküli alcím létrehozása.

Az alkalmazás egy ASP.NET Core webalkalmazás lett. A tesztek futtatására az NUnit csomagot használja, a megjelenítésre pedig a Chart.js csomagot. Az eredményeket egy SQLite adatbázisban tárolja.

\section*{Elért eredmények} % Sorszám nélküli alcím létrehozása.

Egy teljesen működő és tesztelésre alkalmas webalkalmazás. Továbbá egy saját szimmetrikus titkosítási algoritmus C\# megvalósítása.

\section*{Kulcsszavak} % Sorszám nélküli alcím létrehozása.

C\#, ASP.NET Core, NUnit, Chart.js, blokkláncok, kriptográfia

\chapter{Bevezetés} % Sorszámozott fejezet létrehozása.

\section{Motiváció} % Sorszámozott alcím létrehozása.

Mindig is akartam valami maradandót alkotni. Valami olyat, ami nem csak egy kézzel fogható tárgy. Valamit, amire mindig büszkén tudok visszanézi. Még akkor is ha nem a legtökéletesebb munkám lenne. Egy olyan eszmei értéket, amit bárki tud használni kedve szerint. Talán úgy ahogy én elképzeltem. De az is lehet, hogy csak alapul használják és javítanak rajta. Már ez is nagy öröm lenne számomra. Mert látnám, hogy mások szerint is jól sikerült az alkotásom, hiszen mi másért használnak?

Ez volt az hajtó erő, aminek hatására elkezdtem egy saját titkosító algoritmust fejleszteni. Hiszen másoknak is sikerült már előttem. Nem is egy ilyen algoritmus van amit a mindennapi életünkben szinte állandó jelleggel használunk. Mégsem mondjuk soha sem egy algoritmusra, hogy ez a legjobb, vagy éppen az a másik. Ilyen tekintetben szerintem ezek az algoritmusok egy kicsit misztikus dolgok. Mivel nehéz eldönteni, hogy kettő közül melyik a jobb. Mert mindig az a kérdés merül fel, hogy milyen értelemben jobb?

Ebben a szakdolgozatban megpróbálom részletesen összehasonlítani az ismertebb titkosító algoritmusokat különböző szempontok alapján. Lehetőleg ezzel érthetőbbé válik ezen algoritmusok világa egy átlagos ember vagy fejlesztő számára. Továbbá kifejtem a saját algoritmusom fejlesztésének folyamatát. Bele érte minden akadályt és nehézséget is, amibe ennek során beleütköztem.

\section{Kriptográfia} % Sorszámozott alcím létrehozása.

Az emberek szeretik biztonságban tudni a bizalmas dolgaikat. Legyen az egy betöréses támadás vagy egyszerű adatküldés esetén. Ez a probléma ihlette meg a kriptográfiai algoritmusokat. Egy ilyen algoritmus egy nyílt szöveget vár bemenetként, majd egy kulcs segítségével egy titkosított szöveget hozz létre a kapott szövegből. Ez minden kriptográfiai algoritmus alapelve.

Fontos kiemelni, hogy a kriptográfiai algoritmusokat \cite{szteCryptography} több módon tudjuk csoportosítani. Az első lehetőség az, hogy az elvégzett műveletek alapján csoportosítjuk őket. Így megkülönböztetjük a keverő, helyettesítő és produkciós titkosító algoritmusokat.

A keverő algoritmusoknál a titkosított szöveg egyszerűen az eredeti szöveg karaktereinek permutációja. Értelem szerűen ez nem nyújtja a legnagyobb biztonságot. Tekintve, hogy a permutáció lehet akár az eredeti szöveg is. Igaz, ez inkább a rövid szövegek esetén valószínűbb, hogy bekövetkezik.

A helyettesítő algoritmusok a nyílt szöveg karaktereit a titkosított szöveg karaktereivel helyettesítik. Ez történhet egyesével vagy akár blokkokban is. Ezek az algoritmusok már nagyobb biztonságot nyújtanak, mint a keverő algoritmusok. De ez sem egy új keletű dolog, a németek Enigma gépe is ezen az elven alapszik. A ki jártasabb a történelemben az tudja, hogy nagyon sokáig nem tudták feltörni az Enigmával küldött üzeneteket.

A produkciós algoritmusok pedig a kettő elv kombinációján alapszanak. Ezek bizonyultak eddig a leghatékonyabb titkosítási módnak. A napjainkban használt algoritmusok is ide tartoznak. A szakdolgozat keretein belül ilyen algoritmusokról lesz részletesebben szó.

A másik felosztási lehetőség, az algoritmus kulcsa szerint történik. Ebben a felosztásban megkülönböztetjük a szimmetrikus és aszimmetrikus kulcsú algoritmusokat. Vitatható, hogy van egy harmadik csoport is, amiben a hasító algoritmusok vannak.

A szimmetrikus kulcsú algoritmusok onnan kapták a nevüket, hogy ugyan azt a kulcsot használjuk titkosításnál és visszafejtésnél is. Ilyenkor mindkét félnek a birtokában van a kulcs. Ennek az a veszélye, hogy a kulcs kiszivároghat. Ha ez megtörténik, akkor innentől kezdve bári el tudja olvasni a titkosított szövegeket.

Az aszimmetrikus kulcsú algoritmusok ezt úgy próbálták orvosolni, hogy két különböző kulcs van. Az egyiket a szöveg titkosításánál kell használni, a másikat pedig a visszafejtésénél. Így csökken az illetéktelen visszafejtés veszélye, mert az ehhez szükséges kulcsot sosem kell megadni senkinek. Ezeket az algoritmusokat sok helyen használjuk manapság. Például végpontok közötti titkosításos üzenetküldésnél.

Az utolsó csoport a hasító algoritmusok. Ezek oly módon titkosítják a nyílt szöveget, hogy azt nem lehet visszafejteni. Egyértelmű, hogy ezek az algoritmusok nem a legmegfelelőbbek titkosított üzenetek küldésére. Hiszen a fogadó fél nem tudja, hogy mi is volt az üzenet eredeti tartalma. Ezen algoritmusok egyik elterjedt felhasználási módja a digitális aláírás. Melynek segítségével a kapott tartalom eredetiségét lehet ellenőrizni.

A szakdolgozat többi részében konkrét szimmetrikus, aszimmetrikus és hasító algoritmusokról lesz szó. Ezek mind produkciós algoritmusok. A keverő és helyettesítő algoritmusokkal külön nem foglalkozom ezen projekt keretein belül.

\section{Poszt-kvantum kriptográfia} % Sorszámozott alcím létrehozása.

\section{Blokkláncok} % Sorszámozott alcím létrehozása.

\chapter{Felhasznált technológiák} % Sorszámozott fejezet létrehozása.

\section{C\#} % Sorszámozott alcím létrehozása.

Azért választottam a C\# \cite{cSharp} nyelvet, mivel jelenleg is ebben dolgozom a BerényiSoftware Tanácsadó és Szolgáltató Kft. cégnél. Úgy gondolom ebben a nyelvben vagyok a legjártasabb. Már volt szerencsém egy pár alkalmazást elkészíteni benne a cég megbízásából. Ezeknek a pontos mivoltába nem megyek bele, mivel köt a céges titoktartás. A lényeg az, hogy már fejlesztettem benne webalkalmazást, mobilalkalmazást valamint háttér szolgáltatást is. Továbbá ez egyik egyetemi kurzus keretein belül már egy egyszerű játékot is készítettem. Ezért tartom számomra a legalkalmasabb nyelvnek bármilyen jellegű alkalmazás fejlesztésére.

A szakdolgozatot a .NET 8-as verziójában írtam. Amikor elkezdtem a fejlesztést, akkor még ez volt a legfrissebb verzió. Azóta már megjelent a 9-es verzió is. Viszont nem láttam különösebb indokot, hogy átvigyem a projektet az új verzióra. Ezért maradt a 8-as verzión.

\section{ASP.NET Core} % Sorszámozott alcím létrehozása.

Tapasztalataim alapján a manapság leginkább használt C\# alapú webalkalmazás típus egyike az ASP.NET Core \cite{aspNetCore}. Ezen belül megkülönböztetjük a Razor és az MVC fajtát, én inkább az MVC-t szertem használni. Nekem ez a tervezi minta jobban átlátható. Mivel a nyelvet már eldöntöttem, ezért itt már nem volt olyan sok választási lehetőségem. De ez egy cseppet sem zavart. Nincs semmi bajom az ASP.NET Core alkalmazásokkal. Szerintem kifejezetten egyszerűen és gyorsan felépíthetőek. Részben ezért is esett a választásom a webalkalmazásra. Úgy gondolom, hogy ha szükség van egy grafikus felhasználói felületre, akkor fejlesztési sebesség szerint a webalkalmazások a leggyorsabban elkészíthetőek.

\section{NUnit} % Sorszámozott alcím létrehozása.

Azt mondanám, hogy a három leggyakrabban használt Unit teszt csomag C\#-ban az NUnit, xUnit.net és az MSTest. A cég ahol dolgozom, az NUnit-ot \cite{nUnit} használja. Ez nagyban befolyásolta a döntésemet ennél a választásnál. Ugyan még nem kellet különösen foglalkoznom vele a céges munkáim során. Akkor is ezt tartottam a legmegfelelőbbnek. Tekintve, hogy bármikor segítséget tudok kérni a cégnél dolgozó tesztelőktől. Valamint nem gondolom, hogy a cég véletlenül használja pont az NUnit-ot. Hiszen létezik más Unit tesztelésre alkalmas csomag is. Visszatekintve nem bántam meg a választásomat. Semmilyen jellegű problémába nem ütköztem a használata során.

\section{Chart.js} % Sorszámozott alcím létrehozása.

A Chart.js \cite{chartJs} egy nagyon népszerű JavaScript csomag. Kiválóan alkalmas rengeteg különféle diagram kirajzolására. Legyen az statikus vagy interaktív, támogatja minden a kétféle grafikonfajtát. Ami még nagyon hasznos tud lenni benne, az a lehetőség, hogy akár több különböző grafikonfajtát is megjelenítsen egyetlen diagramon. Ennek a funkciónak én is nagy hasznát vettem. Szerintem egy kiváló választás bármilyen jellegű webalkalmazásban, ahol diagramok megjelenítése szükséges. A használata is egyszerű, egy kezdő programozó is könnyedén tudja integrálni az alkalmazásába. De ha netán elakadna, a hivatalos oldalukon lévő dokumentáció rendkívül részletes. A fogalmazásmódja érthető, nem túl bonyolított. Példa kódokat is tartalmaz. Nekem nem sikerült semmilyen megoldhatatlan problémába ütköznöm a csomag használata során. Ezért mindenkinek csak ajánlani tudom.

\section{BouncyCastle.Cryptography} % Sorszámozott alcím létrehozása.

Erre a csomagra azért volt szükség, mert az alap .NET környezetben elég kevés titkosító algoritmus van megvalósítva. Ez volt az első alkalom, hogy BouncyCastle-tól \cite{bouncyCastle} használtam egy csomagot. Egy rövid fórum olvasás után kiderítettem, hogy egy elég jól elfogadott csapatról van szó. Valamint nem is csak C\# nyelvre, hanem Java-ra is fejlesztenek csomagokat. A C\# megvalósításuk teljesen jól működik, nem volt vele semmi probléma. A dokumentációjuk viszont lehetne egy kicsit felhasználó barátabb. Nem mindig találtam meg benne amit kerestem. Én azzal a véleménnyel vagyok, hogy ha egy fórumról gyorsabban megtalálom a nekem szükséges információt, akkor a hivatalos dokumentációnak van még hova fejlődnie. Szerencsére csak egy-két dolog volt, amit meg kellet keresnem a dokumentációban. Mindent összegezve, nem egy rossz csomag. Nem bántam meg, hogy ezt választottam.

\section{Git, GitHub, GitLab} % Sorszámozott alcím létrehozása.

Verziókövetésre Git-et \cite{git} használtam. Leginkább azért, mert ez az egyetlen verziókövető amit ismerek és aktívan használok. Az egyetemen is csak ezt kérték mindig tőlünk. Valamint a cégben is csak ezt használjuk. Nem akartam most egy teljesen új verziókövetőt kipróbálni. Ezért maradtam a már jól bevált Git-nél. A projektet feltöltöttem az egyetemi GitLab-ra \cite{gitLab}, hogy a témavezetőm könnyedén hozzáférhessen. A GitLab hibajegy kezelő részét ebben a projektben nem vettem igénybe, mivel egyedül én dolgoztam rajta. A GitFlow-t is feleslegesnek éreztem. A biztonság kedvért a saját GitHub \cite{gitHub} fiókomra is feltöltöttem. Semmi esetre sem szerettem volna elveszíteni az eddigi munkámat. Ezért láttam észerűnek három helyen tartani az egész projekt kódját. Szerencsére nem tapasztaltam hardver problémát a fejlesztés alatt. Egyszer sem kellet a szerveren lévő másolatról újra építeni a projektet. A verziókövetést illetően sem volt semmilyen jellegű problémám.

\chapter{Architekturális terv} % Sorszámozott fejezet létrehozása.

\section{Webalkalmazás} % Sorszámozott alcím létrehozása.

\section{Model-View-Controller} % Sorszámozott alcím létrehozása.

\section{Szeparáció} % Sorszámozott alcím létrehozása.

\section{Adatbázis} % Sorszámozott alcím létrehozása.

\section{Algoritmusok egységesítés} % Sorszámozott alcím létrehozása.

\section{Tesztelt tulajdonságok} % Sorszámozott alcím létrehozása.

\section{Eredmények megjelenítése} % Sorszámozott alcím létrehozása.

\chapter{Megvalósítás} % Sorszámozott fejezet létrehozása.

\section{Az alkalmazás alapjai} % Sorszámozott alcím létrehozása.

\section{Algoritmusok} % Sorszámozott alcím létrehozása.

\section{NUnit tesztek} % Sorszámozott alcím létrehozása.

\section{Tesztesetek automatizálása} % Sorszámozott alcím létrehozása.

\section{Adattárolás} % Sorszámozott alcím létrehozása.

\section{A webalkalmazás} % Sorszámozott alcím létrehozása.

\section{Tesztek integrálása} % Sorszámozott alcím létrehozása.

\section{Eredmények megjelenítése} % Sorszámozott alcím létrehozása.

\chapter{Saját algoritmus} % Sorszámozott fejezet létrehozása.

\section{Elképzelés} % Sorszámozott alcím létrehozása.

\section{A fejlődés folyamata} % Sorszámozott alcím létrehozása.

\section{Jelenlegi állapot} % Sorszámozott alcím létrehozása.

\chapter{Teszteredmények} % Sorszámozott fejezet létrehozása.

\section{Szimmetrikus algoritmusok} % Sorszámozott alcím létrehozása.

\section{Aszimmetrikus algoritmusok} % Sorszámozott alcím létrehozása.

\section{Hasító algoritmusok} % Sorszámozott alcím létrehozása.

\section{A saját algoritmusom} % Sorszámozott alcím létrehozása.

\chapter{Összefoglalás} % Sorszámozott fejezet létrehozása.

\chapter*{Irodalomjegyzék} % Sorszámozott fejezet létrehozása.
\addcontentsline{toc}{chapter}{Irodalomjegyzék} % A fejezet hozzáadása a tartalomjegyzékhez.

\printbibliography[heading=none] % A citerált irodalmak kilistázása.

\chapter*{Nyilatkozat} % Sorszámozott fejezet létrehozása.
\addcontentsline{toc}{chapter}{Nyilatkozat} % A fejezet hozzáadása a tartalomjegyzékhez.

Alulírott Molnár Gábor Ádám, programtervező informatikus BSc szakos hallgató, kijelentem, hogy a dolgozatomat a Szegedi Tudományegyetem, Informatikai Intézet Szoftverfejlesztés Tanszékén készítettem, programtervező informatikus BSc diploma megszerzése érdekében.

Kijelentem, hogy a dolgozatot más szakon korábban nem védtem meg, saját munkám eredménye, és csak a hivatkozott forrásokat (szakirodalom, eszközök stb.) használtam fel.

Tudomásul veszem, hogy szakdolgozatomat a Szegedi Tudományegyetem Diplomamunka Repozitóriumában tárolja.

\vspace{1cm}

{\large Kelt.: 2025. május 13.}

\vspace{0.5cm}
\hfill
\parbox{5cm}{\centering\hrule\vspace{0.3cm} Aláírás}

\end{document} % A dokumentum vége.
